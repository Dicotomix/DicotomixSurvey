%%%%%%%%%%%%%%%%%%%%%%%%%%%%%%%%%%%%%%%%%
% Journal Article
% LaTeX Template
% Version 1.4 (15/5/16)
%
% This template has been downloaded from:
% http://www.LaTeXTemplates.com
%
% Original author:
% Frits Wenneker (http://www.howtotex.com) with extensive modifications by
% Vel (vel@LaTeXTemplates.com)
%
% License:
% CC BY-NC-SA 3.0 (http://creativecommons.org/licenses/by-nc-sa/3.0/)
%
%%%%%%%%%%%%%%%%%%%%%%%%%%%%%%%%%%%%%%%%%

%----------------------------------------------------------------------------------------
%	PACKAGES AND OTHER DOCUMENT CONFIGURATIONS
%----------------------------------------------------------------------------------------

\documentclass[twoside,twocolumn]{article}

\usepackage{blindtext} % Package to generate dummy text throughout this template 

\usepackage[sc]{mathpazo} % Use the Palatino font
\usepackage[T1]{fontenc} % Use 8-bit encoding that has 256 glyphs
\linespread{1.05} % Line spacing - Palatino needs more space between lines
\usepackage{microtype} % Slightly tweak font spacing for aesthetics

\usepackage[french]{babel} % Language hyphenation and typographical rules
\usepackage[utf8]{inputenc}

\usepackage{graphicx}

\usepackage[hmarginratio=1:1,top=32mm,columnsep=20pt]{geometry} % Document margins
\usepackage[hang, small,labelfont=bf,up,textfont=it,up]{caption} % Custom captions under/above floats in tables or figures
\usepackage{booktabs} % Horizontal rules in tables

\usepackage{lettrine} % The lettrine is the first enlarged letter at the beginning of the text

\usepackage{enumitem} % Customized lists
\setlist[itemize]{noitemsep} % Make itemize lists more compact

\usepackage{abstract} % Allows abstract customization
\renewcommand{\abstractnamefont}{\normalfont\bfseries} % Set the "Abstract" text to bold
\renewcommand{\abstracttextfont}{\normalfont\small\itshape} % Set the abstract itself to small italic text

\usepackage{titlesec} % Allows customization of titles
\renewcommand\thesection{\Roman{section}} % Roman numerals for the sections
\renewcommand\thesubsection{\roman{subsection}} % roman numerals for subsections
\titleformat{\section}[block]{\large\scshape\centering}{\thesection.}{1em}{} % Change the look of the section titles
\titleformat{\subsection}[block]{\large}{\thesubsection.}{1em}{} % Change the look of the section titles

\usepackage{fancyhdr} % Headers and footers
\pagestyle{fancy} % All pages have headers and footers
\fancyhead{} % Blank out the default header
\fancyfoot{} % Blank out the default footer
\fancyhead[C]{Running title $\bullet$ May 2016 $\bullet$ Vol. XXI, No. 1} % Custom header text
\fancyfoot[RO,LE]{\thepage} % Custom footer text

\usepackage{titling} % Customizing the title section

\usepackage{hyperref} % For hyperlinks in the PDF

%----------------------------------------------------------------------------------------
%	TITLE SECTION
%----------------------------------------------------------------------------------------

\setlength{\droptitle}{-4\baselineskip} % Move the title up

\pretitle{\begin{center}\Huge\bfseries} % Article title formatting
\posttitle{\end{center}} % Article title closing formatting
\title{Comparaison de la méthode de saisie dichotomique sur le dictionnaire avec les méthodes d'épellation traditionnelles} % Article title
\author{%
\textsc{Équipe Dicotomix}
\normalsize ENS de Lyon \\ % Your institution
\normalsize \href{mailto:dicotomix@ens-lyon.fr}{dicotomix@ens-lyon.fr} % Your email address
%\and % Uncomment if 2 authors are required, duplicate these 4 lines if more
%\textsc{Jane Smith}\thanks{Corresponding author} \\[1ex] % Second author's name
%\normalsize University of Utah \\ % Second author's institution
%\normalsize \href{mailto:jane@smith.com}{jane@smith.com} % Second author's email address
}
\date{} % Leave empty to omit a date
\renewcommand{\maketitlehookd}{%
\begin{abstract}
\noindent Les méthodes traditionnelles permettant de communiquer en se limitant à une entrée de la forme \textit{oui} ou \textit{non} sont basées sur une épellation lettre par lettre. Ici, une approche différente est considérée, où les mots sont épelés un à un par dichotomie sur l'ensemble du dictionnaire français. Cette méthode est ici comparée aux méthodes traditionnelles en vigueur dans le milieu médical. Nous montrons ainsi qu'une recherche dichotomique sur le dictionnaire permet d'écrire plus efficacement, en comparant les résultats obtenus en terme de temps de frappe de ces différentes méthodes. 
\end{abstract}
}

%----------------------------------------------------------------------------------------

\begin{document}

% Print the title
\maketitle

%----------------------------------------------------------------------------------------
%	ARTICLE CONTENTS
%----------------------------------------------------------------------------------------

\section{Introduction}

\lettrine[nindent=0em,lines=3]{L}es méthodes de saisie traditionnelles utilisées dans le milieu médical reposent toutes sur le même principe. Elles consistent à épeler les mots lettre à lettre, en parcourant l'alphabet pour chacun des lettres. Ce type de méthode peut s'avérer frustrant pour l'utilisateur, qui décide parfois même de renoncer à la communication, tant elle peut être fastidieuse. Une autre approche propose d'utiliser plus efficacement une entrée binaire afin d'écrire non pas une lettre à la fois, mais un mot à la fois. Ainsi, nous mettons ici en avant l'intérêt d'une telle méthode, en les comparant sur l'ensemble du dictionnaire.

%------------------------------------------------

\section{Méthodes traditionnelles}

Toutes les méthodes traditionnelles reposent sur une division de l'alphabet en une table en deux dimensions, où la recherche de la lettre voulue se fait en deux temps. Dans un premier temps, on recherche la ligne dans laquelle se situe la lettre voulue, en posant la question ``est-ce que ma lettre est dans la \textit{première} ligne'', ou dans la \textit{seconde}, jusqu'à arriver à la bonne ligne. Ensuite, on itère de la même manière sur chacune des lettres de la ligne jusqu'à trouver la lettre voulue.

Une des méthodes d'épellation les plus répandues utilise l'alphabet EJASINT décrit dans la figure \ref{ejasint}. Les autres méthodes envisagées ici sont décrites en annexe, et reposent toutes sur le même principe, avec une répartition différente de l'alphabet.
\begin{center}
\begin{figure}
  \includegraphics[scale=0.45]{ejasint.jpg}
  \caption{Alphabet EJASINT, majoritairement utilisé pour la saisie lettre à lettre par l'association ALIS}
  \label{ejasint}
\end{figure}
\end{center}

%------------------------------------------------

\section{Méthode dichotomique}

Notre méthode repose sur un parcours du dictionnaire français par dichotomie, en tenant compte des fréquences des mots dans l'usage courant de la langue française. La question posée porte ainsi sur des mots et n'est plus de la forme ``est-ce le bon mot ?'', mais ``est-il avant ou après dans l'alphabet ?''

Une seconde méthode que nous évoquerons est la méthode de l'épellation dichotomique, reprenant le principe de la dichotomie mais cette fois pour une épellation lettre par lettre.


%------------------------------------------------

\section{Résultats}

Nous nous référerons, dans cette partie, aux différentes méthodes au travers des numéros suivants : 0 - recherche dichotomique, 1 - épellation dichotomique, 2 - alphabet EJASINT, les suivants correspondant aux numéros des alphabets en annexe.

\subsection{Répartition de la longueur de frappe}

Le premier axe de comparaison étudié est le nombre de questions nécessaires afin de taper un mot donné (longueur de frappe) et la distribution de cette longueur de frappe pour chacune des méthodes. Ainsi, nous considérons chacun des mots du dictionnaire avec sa fréquence dans l'usage courant de la langue française pour calculer la fréquence d'apparition d'un mot ayant une longueur de frappe donnée, et ce pour chacune des méthodes considérées (cf. figure \ref{distrib}).

\begin{center}
\begin{figure}
  \includegraphics[scale=0.35]{distrib.png}
  \caption{Distribution du temps de frappe en fonction de la longueur des mots pour différentes méthodes. Bleu - recherche dichotomique, orange : épellation dichotomique, vert : alphabet EJASINT, rouge : 3, violet : 4, marron : 5, rose pâle : 6}
  \label{distrib}
\end{figure}
\end{center}

\begin{center}
\begin{figure}
  \includegraphics[scale=0.35]{frappe-mini.png}
  \caption{Distribution du temps de frappe minimal (ie x questions ou plus) en fonction de la longueur des mots. Bleu - recherche dichotomique, orange : épellation dichotomique, vert : alphabet EJASINT, rouge : 3, violet : 4, marron : 5, rose pâle : 6}
  \label{frappe-mini}
\end{figure}
\end{center}

En figure \ref{distrib}, la méthode de recherche dichotomique affiche une fréquence plus élevée pour les mots de longueur de frappe faible, là où toutes les autres méthodes nécessitent parfois énormément plus de questions pour épeler un mot. La figure \ref{frappe-mini} confirme cette tendance, étant donné que la fréquence d'apparition de mots nécessitant au moins 20 coups pour être tapés est presque nulle, contre plus de 50\% pour les autres méthodes.

La méthode d'épelation dichotomique, quant à elle, donne des résultats proches de l'alphabet EJASINT.

\subsection{Longueur de frappe moyenne}

\begin{center}
\begin{figure}
  \includegraphics[scale=0.5]{mean.png}
  \caption{Longueur de frappe moyenne sur l'ensemble du dictionnaire (barre violettes), avec l'écart-type de la distribution (petites barres violettes claires). 0 - recherche dichotomique, 1 - épellation dichotomique, 2 - alphabet EJASINT, 3 - méthode 3, 4 - méthode 4, 5 - méthode 5, 6 - méthode 6}
  \label{mean}
\end{figure}
\end{center}

La longueur de frappe moyenne (cf. figure \ref{mean}) pour chacune des méthodes affiche ainsi une longueur de frappe moyenne significativement plus faible que toutes les autres méthodes considérées, tout en offrant un écart-type plus faible, ce qui signifie que les mots sont en moyenne tapés plus vite, et qu'il est peu probable d'avoir à écrire un mot long à écrire.


%------------------------------------------------

\section{Conclusion}

La méthode de recherche dichotomique utilisée par le projet Dicotomix est donc une méthode permettant, d'un point de vue théorique, d'écrire de façon significativement plus rapide que les méthodes actuelles. Cependant, elle nécessite un coût cognitif plus important, afin de répondre aux questions nécessaires à identifier un mot. 

%----------------------------------------------------------------------------------------
%	REFERENCE LIST
%----------------------------------------------------------------------------------------

% \begin{thebibliography}{99} % Bibliography - this is intentionally simple in this template

%\bibitem[Figueredo and Wolf, 2009]{Figueredo:2009dg}
%Figueredo, A.~J. and Wolf, P. S.~A. (2009).
%\newblock Assortative pairing and life history strategy - a cross-cultural
%  study.
%\newblock {\em Human Nature}, 20:317--330.
 
%\end{thebibliography}
%
%----------------------------------------------------------------------------------------

\end{document}
